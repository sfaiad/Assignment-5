% Options for packages loaded elsewhere
\PassOptionsToPackage{unicode}{hyperref}
\PassOptionsToPackage{hyphens}{url}
%
\documentclass[
]{article}
\usepackage{lmodern}
\usepackage{amssymb,amsmath}
\usepackage{ifxetex,ifluatex}
\ifnum 0\ifxetex 1\fi\ifluatex 1\fi=0 % if pdftex
  \usepackage[T1]{fontenc}
  \usepackage[utf8]{inputenc}
  \usepackage{textcomp} % provide euro and other symbols
\else % if luatex or xetex
  \usepackage{unicode-math}
  \defaultfontfeatures{Scale=MatchLowercase}
  \defaultfontfeatures[\rmfamily]{Ligatures=TeX,Scale=1}
\fi
% Use upquote if available, for straight quotes in verbatim environments
\IfFileExists{upquote.sty}{\usepackage{upquote}}{}
\IfFileExists{microtype.sty}{% use microtype if available
  \usepackage[]{microtype}
  \UseMicrotypeSet[protrusion]{basicmath} % disable protrusion for tt fonts
}{}
\makeatletter
\@ifundefined{KOMAClassName}{% if non-KOMA class
  \IfFileExists{parskip.sty}{%
    \usepackage{parskip}
  }{% else
    \setlength{\parindent}{0pt}
    \setlength{\parskip}{6pt plus 2pt minus 1pt}}
}{% if KOMA class
  \KOMAoptions{parskip=half}}
\makeatother
\usepackage{xcolor}
\IfFileExists{xurl.sty}{\usepackage{xurl}}{} % add URL line breaks if available
\IfFileExists{bookmark.sty}{\usepackage{bookmark}}{\usepackage{hyperref}}
\hypersetup{
  pdftitle={Assignment-5},
  pdfauthor={Sara Faiad},
  hidelinks,
  pdfcreator={LaTeX via pandoc}}
\urlstyle{same} % disable monospaced font for URLs
\usepackage[margin=1in]{geometry}
\usepackage{color}
\usepackage{fancyvrb}
\newcommand{\VerbBar}{|}
\newcommand{\VERB}{\Verb[commandchars=\\\{\}]}
\DefineVerbatimEnvironment{Highlighting}{Verbatim}{commandchars=\\\{\}}
% Add ',fontsize=\small' for more characters per line
\usepackage{framed}
\definecolor{shadecolor}{RGB}{248,248,248}
\newenvironment{Shaded}{\begin{snugshade}}{\end{snugshade}}
\newcommand{\AlertTok}[1]{\textcolor[rgb]{0.94,0.16,0.16}{#1}}
\newcommand{\AnnotationTok}[1]{\textcolor[rgb]{0.56,0.35,0.01}{\textbf{\textit{#1}}}}
\newcommand{\AttributeTok}[1]{\textcolor[rgb]{0.77,0.63,0.00}{#1}}
\newcommand{\BaseNTok}[1]{\textcolor[rgb]{0.00,0.00,0.81}{#1}}
\newcommand{\BuiltInTok}[1]{#1}
\newcommand{\CharTok}[1]{\textcolor[rgb]{0.31,0.60,0.02}{#1}}
\newcommand{\CommentTok}[1]{\textcolor[rgb]{0.56,0.35,0.01}{\textit{#1}}}
\newcommand{\CommentVarTok}[1]{\textcolor[rgb]{0.56,0.35,0.01}{\textbf{\textit{#1}}}}
\newcommand{\ConstantTok}[1]{\textcolor[rgb]{0.00,0.00,0.00}{#1}}
\newcommand{\ControlFlowTok}[1]{\textcolor[rgb]{0.13,0.29,0.53}{\textbf{#1}}}
\newcommand{\DataTypeTok}[1]{\textcolor[rgb]{0.13,0.29,0.53}{#1}}
\newcommand{\DecValTok}[1]{\textcolor[rgb]{0.00,0.00,0.81}{#1}}
\newcommand{\DocumentationTok}[1]{\textcolor[rgb]{0.56,0.35,0.01}{\textbf{\textit{#1}}}}
\newcommand{\ErrorTok}[1]{\textcolor[rgb]{0.64,0.00,0.00}{\textbf{#1}}}
\newcommand{\ExtensionTok}[1]{#1}
\newcommand{\FloatTok}[1]{\textcolor[rgb]{0.00,0.00,0.81}{#1}}
\newcommand{\FunctionTok}[1]{\textcolor[rgb]{0.00,0.00,0.00}{#1}}
\newcommand{\ImportTok}[1]{#1}
\newcommand{\InformationTok}[1]{\textcolor[rgb]{0.56,0.35,0.01}{\textbf{\textit{#1}}}}
\newcommand{\KeywordTok}[1]{\textcolor[rgb]{0.13,0.29,0.53}{\textbf{#1}}}
\newcommand{\NormalTok}[1]{#1}
\newcommand{\OperatorTok}[1]{\textcolor[rgb]{0.81,0.36,0.00}{\textbf{#1}}}
\newcommand{\OtherTok}[1]{\textcolor[rgb]{0.56,0.35,0.01}{#1}}
\newcommand{\PreprocessorTok}[1]{\textcolor[rgb]{0.56,0.35,0.01}{\textit{#1}}}
\newcommand{\RegionMarkerTok}[1]{#1}
\newcommand{\SpecialCharTok}[1]{\textcolor[rgb]{0.00,0.00,0.00}{#1}}
\newcommand{\SpecialStringTok}[1]{\textcolor[rgb]{0.31,0.60,0.02}{#1}}
\newcommand{\StringTok}[1]{\textcolor[rgb]{0.31,0.60,0.02}{#1}}
\newcommand{\VariableTok}[1]{\textcolor[rgb]{0.00,0.00,0.00}{#1}}
\newcommand{\VerbatimStringTok}[1]{\textcolor[rgb]{0.31,0.60,0.02}{#1}}
\newcommand{\WarningTok}[1]{\textcolor[rgb]{0.56,0.35,0.01}{\textbf{\textit{#1}}}}
\usepackage{graphicx,grffile}
\makeatletter
\def\maxwidth{\ifdim\Gin@nat@width>\linewidth\linewidth\else\Gin@nat@width\fi}
\def\maxheight{\ifdim\Gin@nat@height>\textheight\textheight\else\Gin@nat@height\fi}
\makeatother
% Scale images if necessary, so that they will not overflow the page
% margins by default, and it is still possible to overwrite the defaults
% using explicit options in \includegraphics[width, height, ...]{}
\setkeys{Gin}{width=\maxwidth,height=\maxheight,keepaspectratio}
% Set default figure placement to htbp
\makeatletter
\def\fps@figure{htbp}
\makeatother
\setlength{\emergencystretch}{3em} % prevent overfull lines
\providecommand{\tightlist}{%
  \setlength{\itemsep}{0pt}\setlength{\parskip}{0pt}}
\setcounter{secnumdepth}{-\maxdimen} % remove section numbering

\title{Assignment-5}
\author{Sara Faiad}
\date{2/17/2021}

\begin{document}
\maketitle

\hypertarget{assignment-5---creating-an-expository-figure}{%
\section{Assignment 5 - Creating an expository
figure}\label{assignment-5---creating-an-expository-figure}}

\hypertarget{getting-the-data}{%
\subsection{Getting the Data}\label{getting-the-data}}

I created a data folder locally and downloaded ``siscowet.csv''.

\begin{Shaded}
\begin{Highlighting}[]
\CommentTok{# Reading in the data }

\NormalTok{siscowet<-}\StringTok{ }\KeywordTok{read.csv}\NormalTok{(}\StringTok{"data/siscowet.csv"}\NormalTok{)}

\CommentTok{# Checking things out }

\KeywordTok{head}\NormalTok{(siscowet)}
\end{Highlighting}
\end{Shaded}

\begin{verbatim}
##       locID pnldep mesh fishID  sex age len  wgt
## 1 Deer Park  36.74  2.5  19108 <NA>  NA 316  400
## 2 Deer Park  40.09  3.0  19109 <NA>  NA 396  700
## 3 Deer Park  41.46  5.0  19110    M  NA 590 1800
## 4 Deer Park  41.46  5.0  19111    M  NA 516 1500
## 5 Deer Park  43.45  5.5  19112 <NA>  NA 414  800
## 6 Deer Park  45.58  4.0  19113    M  NA 481 1000
\end{verbatim}

\hypertarget{the-plot---exploratory}{%
\subsection{The Plot - Exploratory}\label{the-plot---exploratory}}

\begin{Shaded}
\begin{Highlighting}[]
\CommentTok{# load ggplot2}

\KeywordTok{library}\NormalTok{(ggplot2)}


\CommentTok{# A basic scatterplot showing the relationship between len and age. Points are colored depending on locID. }

\KeywordTok{ggplot}\NormalTok{(}\DataTypeTok{data =}\NormalTok{ siscowet, }\KeywordTok{aes}\NormalTok{(}\DataTypeTok{x=}\NormalTok{age, }\DataTypeTok{y=}\NormalTok{len, }\DataTypeTok{color=}\NormalTok{locID)) }\OperatorTok{+}\StringTok{ }
\StringTok{    }\KeywordTok{geom_point}\NormalTok{(}\DataTypeTok{size=}\DecValTok{6}\NormalTok{) }
\end{Highlighting}
\end{Shaded}

\begin{verbatim}
## Warning: Removed 580 rows containing missing values (geom_point).
\end{verbatim}

\includegraphics{Assignment-5_files/figure-latex/exploratory_plot-1.pdf}

That looks pretty bad, but it's a start. Time to refine.

\hypertarget{refining-the-plot---expository-1.0}{%
\subsection{Refining the Plot - Expository
1.0}\label{refining-the-plot---expository-1.0}}

\begin{Shaded}
\begin{Highlighting}[]
\CommentTok{#PNW color pallette (because I love living in the PNW)}

\KeywordTok{library}\NormalTok{(PNWColors)}

\CommentTok{#Building a scatterplot }

\NormalTok{siscowet_expository_plot<-}\StringTok{ }\KeywordTok{ggplot}\NormalTok{(}\DataTypeTok{data =}\NormalTok{ siscowet, }
       \KeywordTok{aes}\NormalTok{(}\DataTypeTok{x =}\NormalTok{ age, }
       \DataTypeTok{y=}\NormalTok{ len)) }\OperatorTok{+}\StringTok{ }
\StringTok{  }\KeywordTok{geom_point}\NormalTok{(}\KeywordTok{aes}\NormalTok{(}\DataTypeTok{color =}\NormalTok{ locID, }
                 \DataTypeTok{shape =}\NormalTok{ locID, }
                 \DataTypeTok{alpha =}\NormalTok{ locID),}
             \DataTypeTok{size =} \DecValTok{3}\NormalTok{) }\OperatorTok{+}
\StringTok{  }\KeywordTok{scale_color_manual}\NormalTok{(}\DataTypeTok{values =} \KeywordTok{pnw_palette}\NormalTok{(}\StringTok{"Cascades"}\NormalTok{,}\DecValTok{4}\NormalTok{)) }\OperatorTok{+}
\StringTok{  }\KeywordTok{labs}\NormalTok{(}\DataTypeTok{title =} \StringTok{"Siscowet Lake Trout from Michigan waters of Lake Superior"}\NormalTok{,}
       \DataTypeTok{subtitle =} \StringTok{"Length and weight from four locations: Blind Sucker, Deer Park, Grand Marais, and  Little Lack Harbor"}\NormalTok{,}
       \DataTypeTok{x =} \StringTok{"Assigned Ages (years)"}\NormalTok{,}
       \DataTypeTok{y =} \StringTok{"Total Length (mm)"}\NormalTok{,}
       \DataTypeTok{color =} \StringTok{"Location ID"}\NormalTok{,}
       \DataTypeTok{shape =} \StringTok{"Location ID"}\NormalTok{, }
       \DataTypeTok{alpha =} \StringTok{"Location ID"}\NormalTok{) }\OperatorTok{+}
\StringTok{  }\KeywordTok{theme_classic}\NormalTok{()}

\NormalTok{siscowet_expository_plot}
\end{Highlighting}
\end{Shaded}

\begin{verbatim}
## Warning: Using alpha for a discrete variable is not advised.
\end{verbatim}

\begin{verbatim}
## Warning: Removed 580 rows containing missing values (geom_point).
\end{verbatim}

\includegraphics{Assignment-5_files/figure-latex/expository_plot_1.0-1.pdf}

Better, but still not great. Many of the points overlap and adding some
transperancy to the points doesn't really help matters. Additionally,
the colors are not colorblind friendly. I'm going to switch up tactics
and attempt a violin plot.

\hypertarget{violin-plot---expository-2.0}{%
\subsection{Violin Plot - Expository
2.0}\label{violin-plot---expository-2.0}}

\begin{Shaded}
\begin{Highlighting}[]
\KeywordTok{library}\NormalTok{(dplyr)}
\end{Highlighting}
\end{Shaded}

\begin{verbatim}
## 
## Attaching package: 'dplyr'
\end{verbatim}

\begin{verbatim}
## The following objects are masked from 'package:stats':
## 
##     filter, lag
\end{verbatim}

\begin{verbatim}
## The following objects are masked from 'package:base':
## 
##     intersect, setdiff, setequal, union
\end{verbatim}

\begin{Shaded}
\begin{Highlighting}[]
\KeywordTok{ggplot}\NormalTok{(}\DataTypeTok{data =}\NormalTok{ siscowet, }
       \KeywordTok{aes}\NormalTok{(}\DataTypeTok{fill =}\NormalTok{ sex, }\DataTypeTok{y =}\NormalTok{ len, }\DataTypeTok{x =}\NormalTok{ locID)) }\OperatorTok{+}\StringTok{ }
\StringTok{  }\KeywordTok{geom_violin}\NormalTok{(}\DataTypeTok{position=}\StringTok{"dodge"}\NormalTok{) }\OperatorTok{+}
\StringTok{  }\KeywordTok{labs}\NormalTok{(}\DataTypeTok{title=}\StringTok{"Siscowet Lake Trout from Michigan waters of Lake Superior"}\NormalTok{,}
       \DataTypeTok{x=}\StringTok{"Location ID"}\NormalTok{, }
       \DataTypeTok{y =} \StringTok{"Total Length (mm)"}\NormalTok{,}
       \DataTypeTok{color =} \StringTok{"Sex"}\NormalTok{) }\OperatorTok{+}\StringTok{ }
\StringTok{  }\KeywordTok{scale_fill_manual}\NormalTok{(}\DataTypeTok{values =} \KeywordTok{pnw_palette}\NormalTok{(}\StringTok{"Cascades"}\NormalTok{,}\DecValTok{3}\NormalTok{)) }\OperatorTok{+}
\StringTok{  }\KeywordTok{theme_classic}\NormalTok{()}
\end{Highlighting}
\end{Shaded}

\includegraphics{Assignment-5_files/figure-latex/expository_plot_2.0-1.pdf}

I wouldn't say this is the most beautiful thing that I have ever seen,
but I think it's pretty good, especially for never making a violin plot
before! The shape of the violin represents the total length of trout
(mm), separated by sex, from each location. Sex is also reflected by the
color of each violin. Green denotes females, yellow males, and white for
NAs. The colors still appear distinct on the Color Blindness Simulator.

\end{document}
